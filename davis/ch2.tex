\chapter{Programs and Computable Functions}


\section{A Programming Language}
No exercise.



\section{Some Examples of Programs}

\subsection{}
\begin{center}
\begin{tabular}{l}
  $ Z \gets Z + 1 $ \\
  $ Z \gets Z + 1 $ \\
  $ Z \gets Z + 1 $ \\
  $ Y \gets X \cdot Z $ \\
\end{tabular}
\end{center}


\subsection{}
Admit.


\subsection{}
\begin{center}
\begin{tabular}{ll}
        & $ Z \gets X $ \\
        & $ Y \gets Y + 1 $ \\
  $[A]$ & IF $Z \neq 0$ GOTO $B$ \\
        & GOTO $E$ \\
  $[B]$ & $ Z \gets Z - 1 $ \\
        & IF $Z \neq 0$ GOTO $C$ \\
        & $ Y \gets Y - 1 $ \\
        & GOTO $E$ \\
  $[C]$ & $ Z \gets Z - 1 $ \\
        & IF $X \neq 0$ GOTO $A$ \\
\end{tabular}
\end{center}


\subsection{}
\begin{center}
\begin{tabular}{ll}
        & $ Z \gets X $ \\
        & $ Y \gets Y + 1 $ \\
  $[A]$ & IF $Z \neq 0$ GOTO $B$ \\
        & GOTO $E$ \\
  $[B]$ & $ Z \gets Z - 1 $ \\
        & IF $Z \neq 0$ GOTO $C$ \\
        & GOTO $B$ \\
  $[C]$ & $ Z \gets Z - 1 $ \\
        & IF $X \neq 0$ GOTO $A$ \\
\end{tabular}
\end{center}


\subsection{}
\begin{center}
\begin{tabular}{ll}
  $[A]$ & IF $X_{1} \neq 0$ GOTO $B$ \\
        & IF $X_{2} \neq 0$ GOTO $E$ \\
        & $ Y \gets Y + 1 $ \\
        & GOTO $E$ \\
  $[B]$ & IF $X_{2} \neq 0$ GOTO $C$ \\
        & GOTO $E$ \\
  $[C]$ & $ X_{1} \gets X_{1} - 1 $ \\
        & $ X_{2} \gets X_{2} - 1 $ \\
        & GOTO $A$ \\
\end{tabular}
\end{center}


\subsection{}
\begin{center}
\begin{tabular}{ll}
  $[A]$ & $ Y \gets Y + 1 $ \\
        & $ Z_{1} \gets Y $ \\
        & $ Z_{2} \gets Y $ \\
        & $ Z_{3} \gets Z_{1} \cdot Z_{2} $ \\
        & $ Z_{4} \gets X $ \\
  $[B]$ & IF $Z_{3} \neq 0$ GOTO $C$ \\
        & GOTO $A$ \\
  $[C]$ & IF $Z_{4} \neq 0$ GOTO $D$ \\
        & $ Y \gets Y - 1 $ \\
        & GOTO $E$ \\
  $[D]$ & $ Z_{3} \gets Z_{3} - 1 $ \\
        & $ Z_{4} \gets Z_{4} - 1 $ \\
        & GOTO $B$ \\
\end{tabular}
\end{center}


\subsection{}
\begin{center}
\begin{tabular}{ll}
            & $ Z_{1} \gets X_{1} $ \\
            & $ Z_{2} \gets X_{2} $ \\
  $[A_{1}]$ & IF $X_{1} \neq 0$ GOTO $A_{2}$ \\
            & $ Y \gets X_{2} $ \\
            & GOTO $E$ \\
  $[A_{2}]$ & IF $X_{2} \neq 0$ GOTO $B$ \\
            & $ Y \gets X_{1} $ \\
            & GOTO $E$ \\
  $[B]$     & $ X_{1} \gets X_{1} - 1 $ \\
            & $ X_{2} \gets X_{2} - 1 $ \\
            & GOTO $C_{1}$ \\
  $[C_{1}]$ & IF $X_{1} \neq 0$ GOTO $C_{2}$ \\
            & $ X_{1} \gets Z_{1} $ \\
            & $ Z_{2} \gets X_{2} $ \\
            & GOTO $A_{1}$ \\
  $[C_{2}]$ & IF $X_{2} \neq 0$ GOTO $B$ \\
            & $ X_{2} \gets Z_{1} $ \\
            & $ Z_{1} \gets X_{1} $ \\
            & GOTO $A_{1}$ \\
\end{tabular}
\end{center}




\section{Snytax}

\subsection{}
Admit.


\subsection{}
Admit.


\subsection{}
\begin{center}
\begin{tabular}{ll}
  $[A]$ & $ X \gets X - 1 $ \\
        & IF $X \neq 0$ GOTO $A$ \\
\end{tabular}
\end{center}
This program does not work for n = 0.




\section{Computable Functions}

\subsection{}
$ \psi_{\mathscr{P}}^{(1)}(x) $ is undefined for all $x$.


\subsection{}
\begin{displaymath}
\psi_{\mathscr{P}}^{(1)}(x) = \left\{
\begin{array}{ll}
  \uparrow & \mathrm{if}\ x = 0 \\
  x        & \mathrm{if}\ x > 0 \\
\end{array}
\right.
\end{displaymath}


\subsection{}
\begin{displaymath}
\psi_{\mathscr{P}}^{(1)}(x) = 0
\end{displaymath}


\subsection{}
\[ \psi_{\mathscr{P}}^{(1)}(r_{1}) = r_{1} \]
\[ \psi_{\mathscr{P}}^{(2)}(r_{1},r_{2}) = \left\{
   \begin{array}{ll}
     r_{1}        & \mathrm{if}\ r_{2} = 0 \\
     r_{1}+2r_{2} & \mathrm{else} \\
   \end{array}
   \right.
\]
\[ \psi_{\mathscr{P}}^{(2)}(r_{1},r_{2},r_{3}) = \left\{
   \begin{array}{ll}
     r_{1}        & \mathrm{if}\ r_{2} = 0 \\
     r_{1}+2r_{2} & \mathrm{else} \\
   \end{array}
   \right.
\]


\subsection{}
Since $f$ is a partially computable function, there must be an program
$\mathscr{P}$ that compute $f$. Suppose the length of $\mathscr{P}$ is
$m$, then by substituting the subscript of local variables or label all
at once, we can obtain a variant that do the same things, without changing
the length. There are infinitely many variant because we have infinite
local variables and labels.


\subsection{}
\begin{enumerate}
\item
Program contains $k$ same instructions \[ Y \gets Y + 1 \] followed by
\[ \mathrm{GOTO}\ E \] computes $f_{k}$.

\item Admit.
\item Admit.

\item
Suppose a straightline program $\mathscr{P}$ computes the function
$ f(x) = x + 1 $, whose length is $k$. Then $ \psi_{\mathscr{P}}^{(1)}(k) = k+1 $,
which is contradict to (b).
\end{enumerate}


\subsection{}
Abort.

\subsection{}
Admit.



\section{More about Macros}

\subsection{}
Admit.

\subsection{}
Admit.

\subsection{}
\begin{center}
\begin{tabular}{l}
  $ Z \gets g(X) $ \\
  $ Y \gets f(Z) $ \\
\end{tabular}
\end{center}


\subsection{}
\begin{center}
\begin{tabular}{ll}
        & $ Z_{1} \gets X_{1} $ \\
        & $ Z_{2} \gets X_{2} $ \\
  $[A]$ & IF $Z_{1} \neq 0$ GOTO $B$ \\
        & $ Y \gets Y + 1 $ \\
        & GOTO $E$ \\
  $[B]$ & IF $Z_{2} \neq 0$ GOTO $C$ \\
        & GOTO $E$ \\
  $[C]$ & $ Z_{1} \gets Z_{1} - 1 $ \\
        & $ Z_{2} \gets Z_{2} - 1 $ \\
        & GOTO $A$ \\
\end{tabular}
\end{center}


\subsection{}
\begin{center}
\begin{tabular}{ll}
        & $ Y \gets X_{1} + X_{2} $ \\
  $[A]$ & IF $ P (Y) $ GOTO $E$ \\
        & GOTO $A$ \\
\end{tabular}
\end{center}


\subsection{}
\begin{center}
\begin{tabular}{ll}
  $[A]$ & IF $ P (Z) $ GOTO $C$ \\
  $[B]$ & $ Z \gets Z + 1 $ \\
        & GOTO $A$ \\
  $[C]$ & $ X \gets X - 1 $ \\
        & IF $ X \neq 0 $ GOTO $B$ \\
        & $ Y \gets Y + 1 $  \\
        & GOTO $E$ \\
\end{tabular}
\end{center}


\subsection{}
Firstly show the predicate $ x_{1} = x_{2} $ is computable:
\begin{center}
\begin{tabular}{ll}
        & IF $ X_{1} \le X_{2} $ GOTO $A$ \\
        & GOTO $E$ \\
  $[A]$ & IF $ X_{2} \le X_{1} $ GOTO $B$ \\
        & GOTO $E$ \\
  $[B]$ & $ Y \gets Y + 1 $ \\
        & GOTO $E$ \\
\end{tabular}
\end{center}
Then $\pi^{-1}$ is computed by following program:
\begin{center}
\begin{tabular}{ll}
  $[A]$ & $ Z \gets \pi(Y)$ \\
        & IF $ X = Z $ GOTO $E$ \\
        & $ Y \gets Y + 1 $ \\
        & GOTO $A$ \\
\end{tabular}
\end{center}


\subsection{}
Admit.

\subsection{}
Admit.


\subsection{}
In $\mathscr{S}'$ we also have $ V \gets V + 1$, and by using macro we can 
simulate other 2 forms of instruction.
\begin{enumerate}
\item 
find a unused variable $ Z_{m} $ (which initially has the value 0),
then IF $ V \neq Z_{m} $ GOTO $L$ is actually IF $ V \neq 0 $ GOTO $L$.

\item 
following program computes $ f(x) = x - 1 $:
\begin{center}
\begin{tabular}{ll}
        & IF $ X \neq 0 $ GOTO $A$ \\
        & GOTO $E$ \\
  $[A]$ & $ Z \gets Z + 1 $ \\
        & IF $ X \neq Z $ GOTO $B$ \\
        & GOTO $E$ \\
  $[B]$ & $ Y \gets Y + 1 $ \\
        & GOTO $A$ \\
\end{tabular}
\end{center}

\end{enumerate}
