\chapter{A Universal Program}


\section{Coding Programs by Numbers}

\subsection{}
Admit.

\subsection{}
\begin{align*}
  575+1 & = 2^6 \cdot 3^2 = [6,2] \\
  6     & = <0,3> = <0,<2,0>> \\
  2     & = <0,1> = <0,<1,0>> \\
\end{align*}
Thus, program $\mathscr{P}$ is:
\begin{align*}
  Y \gets Y - 1 \\
  Y \gets Y + 1
\end{align*}



\section{The Halting Problem}

\subsection{}
Admit.

\subsection{}
If $ \mathrm{\overline{HALT}}(x,y) $ is computable, then so is
$ \mathrm{HALT}(x,y)$ :
\begin{align*}
  & \mathrm{IF}\ \mathrm{\overline{HALT}}(X_1,X_2)\neq 0  \ \mathrm{GOTO}\ E \\
  & Y \gets Y + 1
\end{align*}


\subsection{}
\[ \mathrm{HALT}(x,y) = \mathrm{HALT}^1(<x,y>) \]


\subsection{}
$ \mathrm{HALT}(x,y) $ is total and $ \mathrm{HALT}(x,y) \le 1 $ for all $ x,y $.
But it is not computable.


\subsection{}
Suppose $\#(\mathscr{P}) = n $.
Let program $\mathscr{Q}$ be $\mathscr{P}$ followed instruction:
\[ [E] \text{ IF } Y \neq 0 \text{ GOTO } E \]
then
\[ \psi_{\mathscr{Q}}(x) = \sim \mathrm{HALT}(x,x) \]
and
\[ \#(\mathscr{Q}) = n\cdot P_{Lt(n)+1}^{<5,<7,0>>} .\]
So 
\[ \psi_{\mathscr{P}}(\#(\mathscr{Q})) 
   \Leftrightarrow \mathrm{HALT}(\#(\mathscr{Q}, \#(\mathscr{Q})
   \Leftrightarrow \sim \mathrm{HALT}(\#(\mathscr{Q}, \#(\mathscr{Q}) .\]


\subsection{}
Goldbach's conjecture is either true or false, and has nothing to do with
the input $x$. If it's true, then $ f(x) = x $; otherwise, $ f(x) = 0 $.
In both cases, $f(x)$ is primitive recursive.



\section{Universality}

\subsection{}
Admit.

\subsection{}
\begin{enumerate}
  \item 
  \begin{align*}
    & Z \gets \Phi(X,X) \\
    & Y \gets Y + 1
  \end{align*}
  
  \item 
  \begin{align*}
    & \text{IF } X = a_{1} \text{ GOTO } E \\
    & \text{IF } X = a_{2} \text{ GOTO } E \\
    & \cdots \\
    & Z \gets \Phi(X,X) \\
    & Y \gets Y + 1
  \end{align*}
  
  \item 
  Suppose program $\mathscr{P}$ begin with following instruction:
  \[ [A] \text{ IF } X \neq 0 \text{ GOTO } A \]
  then $\mathscr{P}$ is undefined for all $ x \neq 0 $. Since
  $ \#(\mathscr{P}) > 0 $, so $ \Phi(\#(\mathscr{P}), \#(\mathscr{P}))\uparrow $
  for such $\mathscr{P}$. Take
  \[ B = \{b \ |\ (b)_1 = <1,<3,1>> \} ,\]
  Then $ H_3(x) $ is partially computable:
  \begin{align*}
    & \text{IF } (X)_1 = <1,<3,1>> \text{ GOTO } E \\
    & \cdots \\
    & Z \gets \Phi(X,X) \\
    & Y \gets Y + 1
  \end{align*}
  
  \item 
  \[ C = \{ x\ |\ \Phi(x,x)\uparrow \} \]
  Then
  \[ \begin{split}
    H_4(x) & \Leftrightarrow \Phi(x,x)\downarrow \\
           & = \mathrm{HALT}(x,x)
  \end{split} \]
  which is not computable.
\end{enumerate}


\subsection{}
If $\mathscr{P}$ computes $\Phi^{(1)}(x,y) $, then
\[\begin{split}
  H_{\mathscr{P}}(x,y) & \Leftrightarrow \Phi^{(1)}(x,y) \text{ halts on inputs } x,y \\
                       & = \mathrm{HALT}(x,y)
\end{split}\]


\subsection{}
\[ f(x_1,x_2,\dots,x_n) = 
   (r(\mathrm{SNAP}^{(n)}(x_1,x_2,\dots,x_n,\#(\mathscr{P}),g(x_1,x_2,\dots,x_n))))_1 \]


\subsection{}
Abort.

\subsection{}
Abort.